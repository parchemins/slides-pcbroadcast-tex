\pdfpageattr {/Group << /S /Transparency /I true /CS /DeviceRGB>>}
\documentclass[10pt, xcolor={usenames, dvipsnames}]{beamer}

%\PassOptionsToPackage{cmyk}{xcolor}
%\usepackage[cmyk]{xcolor}
%\documentclass[handout, 10pt, xcolor={usenames, dvipsnames}]{beamer}

%\usepackage{etex}
\usepackage[french, english]{babel}
\usepackage[utf8]{inputenc}
\usepackage[T1]{fontenc}
\usepackage{upgreek,textgreek}
%\usepackage{textcomp}
\usepackage{animate}

\usepackage{pifont}
\newcommand{\cmark}{\ding{51}}
\newcommand{\xmark}{\ding{55}}

%\usepackage{amsfonts}
\usepackage{amsmath}
\usepackage{amssymb}
%\usepackage{mathtools}

\usepackage{graphicx}
\usepackage{hyperref}
\usepackage[style=alphabetic, autocite=inline, firstinits=true, maxnames=2]{biblatex}
\bibliography{bibliographie}
\renewcommand*{\bibfont}{\small}


\usepackage{tabularx, booktabs}
\usepackage{tikz}
\definecolor{light-gray}{gray}{0.65}
\tikzset{fade on/.code={\only<#1>{\color{light-gray}}}}
\tikzset{hide on/.code={\only<#1>{\color{white}}}}
\tikzset{bold on/.code={\only<#1>{\bfseries}}}
\tikzset{
  opinvisible/.style={opacity=0.2},
  visible on/.style={alt={#1{}{opinvisible}}},
  alt/.code args={<#1>#2#3}{%
    \alt<#1>{\pgfkeysalso{#2}}{\pgfkeysalso{#3}} % \pgfkeysalso doesn't change the path
  },
}

\usetikzlibrary{plotmarks,shapes}
\usepackage{colortbl}
\usepackage{ulem}
\usepackage{multicol}
\usepackage[overlay]{textpos}
\usepackage{multirow}
\usepackage{ragged2e}
\usepackage{rotating}
\usepackage{fancybox}
\usepackage{ulem}
\usepackage{overpic}
\usepackage{enumerate}
\usepackage{xfrac}
\usepackage{pgfplots}

\usepackage{transparent}

\usepackage{epstopdf}
\usepackage{epsfig}

\newcommand{\REF}{\textcolor{purple}{REF}}
\newcommand{\YES}[1]{\textcolor{green}{#1}}
\newcommand{\NO}[1]{\textcolor{red}{#1}}

\usetheme{metropolis}

\title{Breaking the Scalability Barrier of Causal Broadcast for Large and Dynamic Systems}
\author{Brice N\'edelec, Pascal Molli, and \textbf{Achour Most{\'e}faoui}}
\date{Workshop O'Browser 2018}
\institute{University of Nantes, LS2N}



\begin{document}

\maketitle

\begin{frame}{Introduction}
  Causal broadcast is the core of many distributed applications such as
  distributed social networks, distributed collaborative softwares, or
  distributed data stores.
\end{frame}

\begin{frame}{Large scale static systems: \YES{}}
\end{frame}

\begin{frame}{But dynamic systems: \NO{}}
\end{frame}

\begin{frame}{Problem}
  Keeping the complexity of large scale static systems in dynamic settings where
  processes can join, leave, or self-reconfigure at any time
\end{frame}

\begin{frame}{Proposal: PC-broadcast}
  We define safe links, and processes use all and only safe links for causal broadcast.
\end{frame}

\begin{frame}{Network conditions may lead to unbounded space consumption}
\end{frame}

\begin{frame}{Handling failures}
\end{frame}

\begin{frame}{Experimentation: small negative impact on the overlay}
\end{frame}

\begin{frame}{Conclusion}
\end{frame}

\end{document}
